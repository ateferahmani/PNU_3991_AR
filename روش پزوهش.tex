\documentclass[a4paper]{book}
\begin{document}
All uses of this research data will remain anoonymous and confidentiolity will be maintained by use of pseudonyms.No real nams will be mentioned in reports or publications related to the data.
Copies of the results of this research will not be immediately available to studets.As the studies using this data are completed the results will be published and may be obtioned on reqquest.
As an industry based participant /student/teaching assistant/faculty member using e-soft-ware in a course Igive my permissien for the computer generated usag data conference transcript data and virtuaal artifacts data collected by the conferencing software to be used for research.
I undrestand that I maay withdraw my permission at any time. I also undrestand that I may register any concern or complaint I might have about the experiment with the dean of this university[name,email,address,and phon number of dean].
Name:
Email:
Data:
If you would like to comment on your experiance as a participant in this project you can use the subject Feedback Form.

note on this consent fom that the purpose of the study is identified paarticipant expectations are outlined confidentiolity and anonymity are explained and a contact for further questions is identified. While this consent form was designed to be completed either as apaper-based form or as an online form email is also often used for obtioning consent.Generally email is the most cost-effective and timely way to communicaates with potentiol respondents and it is also more personal than the pointer to a website thaat explain details of the study.morever requesting thaat the respondents email program send a receipt when the recipient opens the email can enhance the urgency of the message.an automaated request from the e-reseaarches email proogram will trigger a request for permission to email this notification of the potentiol respondent and will likely result in increased response rates.Dillman (2000)recommends sending such a request for participaations prior to transmission of the actual reseaarch instrument.This poolietly informs potentiol respondents of their opportunity to participat without overloading their mailbox through distribution of email.
We can also see from the above example that of consent forms for Net-based research is essentially very similar to other traditional forms of research.Specifically because the medium lends itself to supporting fluid communities that are often of flux and transient obtaining consent which typically includes dissemination of results and/or member checks and debrifings can be a challenge for the e-researcher.King observes "given that cyberspace are tenuous conglomerations of individual where  
membership changes freqquentely even asking for permission from the group is often no assuraance that all members studies will feel they gave informed consent."Research in these communitias requies the development of unique guidelines that are functional within this emergingcontext.
\newpage
Sharf(1999)developed the following set of four ethical guidelines for researchers working with these types of informal mail lists or Usenet groups.
1.The researcher should be sure that the research has the potential to benefit both the researcher and the group that is the subject of the research.
2.The researcher should make clear his or her identity role purpose and intention.
3.The researcher should obtion explicit permission befor directly quoting words of anyone in a publication.
4.The researcher should seek ways to maintain aan oppeness and communication with all participants.
since a list owner or moderator manages most Usenet groups aand active mailing list the approopriae place to begine the type of communications outlined by Sharf is with the list owner.Collaborating with the list owner will often result in higher partcipaation rates and the study may be able to be refined so as to better serve the needs of the list owner and participating members of the online community as well as the e-reseaarcher.We often analyze the transcripts of educational classes delivered online.
During one such reseaarch project we attempted to study the computer conference use by medical student internet.Theconferencing was designed to provide both support and guidence from campus-based instructors.As is customery we field a request for ethical clearance of this reseaarch with the universitys research ethics review communitee.After approval we emailed from to all interns wich grainted us permissin to analyze the transcripts prodused during the internship.In this caase a single student refused to sign the form argiung that the researchers review of their conversations a breech of the privacy that they felt existed within this private conference group.There was little we could do but to abort the research project aat that point.since arguable even this students comments requier a review and analyze of the transcript.More typicaly we find that acquring in form consent forms returned either electronically or paper format is easily obtioned from the majority of students.We are careful to fallow normal protocol of explaining that allowing for transcript review is not compolsory that participants may withdraw from the study that individual privacy will be saafe guarded and that there are no apparent risks to the participants.However we often have problem when in dividuaal student they do not refuse to participate and thus analysis of any informatin from these individuals should be percived as unethical?Or haas the email been lost in transport?Or have the students dropped out of the course?this creats a problem with limited solutions for the e-researchrs.
The above-noted problems aside it is usual for a high proportion of participant to explicity refuse to participate.A solution may be found by contacting the nonrespondents through telephone or reguler mail but often such information is deemed o be private and not available to the researcher.A secend aalternativ is to exclud from 
\newpage
the study only those participants who explicipty refuse participation through notification to the researcher.In this case those who respond favorably or who do not respond aat all are assumed to have given permission for participation.Aa third alternativ is to selectively delet from the analyzist the messege posted by those who have ficalt task however haas often components of previos massege are quoted in subsequent message making extraction of particulor messages very challengin.The safe option to this dilemma is to fallow the third course and eliminate all the comment except those who have explicitly given their permissin to participat.In practice extra effors using the first alternativ and rigorously persuing a nonresponse is probably the best use of e-research time. some researchers avoid this dilemma altogether by public nouncing their perticipation to the online group but only requesting formal permissin from thoose whose writing they are intending to directly qoute from in published result.this issu of informed consent for analyzist of online activity will likely remain for some time.
obtioning cosent electronically
Uniquo ethical issues related to Net-based research also revolve around identifing research subject and their responsed. this isses are not uniqu to Net-based research but tend to because more example when using the internet and hence more problematic.For example the proces of consent usauly involves signing a consent form in wich the researchers outlines the relevant components of the research.Normally a signatior authenticates consent however many potential respondents do not use digital signatur that insure encryption and authenticatian.Thus e-reseaarchers may collect emails or web forms that in a technical sens.do not have the legal weight of a signed consent form.However unless the reseaarcher has reason to belive that the participant has an incentiv to misrepresent him/herself these forms are generally deemed to be accetable notic of consent.E-researchers are encouraged to use authentication software both participants and themselves.software also effectively eliminates third paarty interferenc.These services are provided at relatively law cost through certificates authenticatian and public key infrasturactur firms such as www.verisign.com.
E-researchers who abtion consent over the Net also have to be aware of the risk of possiby attraacting vulnerable populations to their study.AAs Roberts points out important demographic details such as age maybe concealed by potentiol participat.This may lead to vulnerable pupulations.being and included in a study without the researchers knaledge.Schrum maintains that this aloon present serioos problems in obtaining some deggree of informed consent and consider this to be the most difficult eithhical issue of online research.While the e-researcher will need to aacknaledge this as a loooming posobolity to cundocting Net-based research.    
 
\end{document}



















